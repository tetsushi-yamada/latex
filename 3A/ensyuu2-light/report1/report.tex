\documentclass[uplatex]{jsarticle}
%数式
\usepackage{amsmath,amssymb,bm}
%箇条書き
\usepackage{enumerate}
%画像
\usepackage[dvipdfmx]{graphicx}

\pagestyle{empty}%ページ番号が欲しい場合は不要

%−−−−−−−−−−−−−−以下変更不可−−−−−−−−−−−−−−−−−

\makeatletter

\def\department#1{\def\@department{#1}}
\def\id#1{\def\@id{#1}}

\def\@maketitle{
\begin{center}
  {\LARGE \@title \par}%タイトル
  \vskip 1em
  {\large \@date}%日付
  \vskip 0.5em
  {\large \@department}%科類
  \vskip 0.5em
  {\large \@id}%学籍番号
  \hspace{0.5em}
  {\large \@author}%著者
\end{center}
}

\makeatother

%−−−−−−−−−−−−−−以下変更可−−−−−−−−−−−−−−−−


%タイトルの表示設定

\title{ウィーナー・フィルタの導出}%タイトル
\date{\today}%日付 「\today」のままなら今日の日付が出力されます。
\department{工学部計数工学科 システム情報工学コース}%学部学科
\id{03-240641}%学籍番号
\author{山田哲士}%名前


%ここから文書
\begin{document}
\maketitle

未知の原信号 \( X(\omega) \) にフィルタ \( H(\omega) \) が掛かり、さらにノイズ \( N(\omega) \) が加わった劣化信号 \( Y(\omega) \) が得られたとする。
\begin{equation}
Y(\omega) = H(\omega) X(\omega) + N(\omega) \tag{1}
\end{equation}

ノイズが定常で信号と無相関な場合、劣化信号にウィーナー・フィルタ \( G(\omega) \) を掛けると、復元誤差(誤差パワー・スペクトルの期待値)を最小にできる。
\begin{equation}
E\left[ \left| X(\omega) - G(\omega) Y(\omega) \right|^2 \right] \to \min \tag{2}
\end{equation}

煩雑なので、以下では \( X(\omega) \) などの \( (\omega) \) を省略する。式 (2) を
\begin{equation}
l = E\left[ | X - G Y |^2 \right] \tag{3}
\end{equation}
と表すことにして、式 (1) を代入して展開していく。
\begin{align*}
l &= E\left[ | X - G Y |^2 \right] \\
&= E\left[ | X - G(H X + N) |^2 \right] \\
&= E\left[ | (1 - G H) X - G N |^2 \right]
\end{align*}

一般に \( |z|^2 = z z^* \) だから、
\begin{align*}
l &= E\left[ \left\{ (1 - G H) X - G N \right\} \left\{ (1 - G H)^* X^* - G^* N^* \right\} \right] \\
&= (1 - G H)(1 - G H)^* E[X X^*] - G (1 - G H)^* E[N X^*] - G^* (1 - G H) E[X N^*] + |G|^2 E[N N^*]
\end{align*}

ここで、\( G \) と \( H \) は確定値なので期待値演算 \( E[\ ] \) の外に出せる。また、原信号 \( X \) とノイズ \( N \) が無相関であるとすると、\( E[N X^*] = E[N^* X] = 0 \) となる。

原信号 \( X \) のパワー・スペクトルを
\begin{equation}
P_S = E[ |X|^2 ] = E[ X X^* ] \tag{9}
\end{equation}
ノイズ \( N \) のパワー・スペクトルを
\begin{equation}
P_N = E[ |N|^2 ] = E[ N N^* ] \tag{10}
\end{equation}
とする。以上を踏まえて、式を整理すると:
\begin{equation}
l = (1 - G H)(1 - G H)^* P_S + |G|^2 P_N \tag{12}
\end{equation}

展開すると、
\begin{align*}
l &= P_S - (G H + G^* H^*) P_S + |G|^2 |H|^2 P_S + |G|^2 P_N \\
&= P_S - 2 \Re \left[ G H \right] P_S + |G|^2 \left( |H|^2 P_S + P_N \right)
\end{align*}
ここで、\( \Re \left[ G H \right] \) は \( G H \) の実部を表す。

誤差 \( l \) を \( |G| \) の二次式として整理すると、
\begin{equation}
l = a |G|^2 + b G + b^* G^* + c \tag{17}
\end{equation}
ただし、
\begin{align*}
a &= |H|^2 P_S + P_N \\
b &= - H P_S \\
c &= P_S
\end{align*}

一般に、\( a > 0 \) のとき、二次形式 \( a |G|^2 + b G + b^* G^* + c \) は次のように変形できる。
\begin{equation}
a |G|^2 + b G + b^* G^* + c = a \left| G + \frac{b^*}{a} \right|^2 + c - \frac{|b|^2}{a} \tag{22}
\end{equation}

したがって、誤差 \( l \) は次のように表される。
\begin{equation}
l = \left( |H|^2 P_S + P_N \right) \left| G - \frac{H^* P_S}{|H|^2 P_S + P_N} \right|^2 + \frac{P_S P_N}{|H|^2 P_S + P_N} \tag{29}
\end{equation}

このとき、誤差 \( \varepsilon \) が最小になるのは次式を満たすときである。
\begin{equation}
G = \frac{H^* P_S}{|H|^2 P_S + P_N} \tag{30}
\end{equation}

従って、
\begin{equation}
  \varepsilon = \frac{P_N}{P_S} \tag{31}
\end{equation}

とすれば良いこととなる。\\

参考文献 \\
https://www.allisone.co.jp/html/Notes/DSP/Filter/Wiener-filter/index.html

\end{document}
