\documentclass[uplatex]{jsarticle}
%数式
\usepackage{amsmath,amssymb,bm}
%箇条書き
\usepackage{enumerate}
%画像
\usepackage[dvipdfmx]{graphicx}

\pagestyle{empty}%ページ番号が欲しい場合は不要}

%−−−−−−−−−−−−−−以下変更不可−−−−−−−−−−−−−−−−−

\makeatletter

\def\department#1{\def\@department{#1}}
\def\id#1{\def\@id{#1}}

\def\@maketitle{
\begin{center}
  {\LARGE \@title \par}%タイトル
  \vskip 1em
  {\large \@date}%日付
  \vskip 0.5em
  {\large \@department}%科類
  \vskip 0.5em
  {\large \@id}%学籍番号
  \hspace{0.5em}
  {\large \@author}%著者
\end{center}
}

\makeatother

%−−−−−−−−−−−−−−以下変更可−−−−−−−−−−−−−−−−


%タイトルの表示設定

%{}内を書き換えてください。
\title{CTの原理とフーリエスライス定理による説明}%タイトル
\date{\today}%日付 「\today」のままなら今日の日付が出力されます。
\department{工学部計数工学科 システム情報工学コース}%学部学科
\id{03-240641}%学籍番号
\author{山田哲士}%名前


%ここから文書
\begin{document}
\maketitle

\section*{1. 投影データの取得}

まず、被写体の断面を表す2次元関数を $f(x, y)$ とする。CTスキャンでは、様々な角度 $\theta$ からX線を照射し、その投影データ(ラドン変換)を取得する。投影データ $p_\theta(s)$ は、線 $L$ 上での $f(x, y)$ の積分で表される。

\[
p_\theta(s) = \int_{-\infty}^{\infty} \int_{-\infty}^{\infty} f(x, y) \delta(s - x\cos\theta - y\sin\theta) \, dx \, dy
\]

ここで、$\delta$ はディラックのデルタ関数であり、積分は線 $L: x\cos\theta + y\sin\theta = s$ 上で行われる。

\section*{2. フーリエスライス定理の定式化}

フーリエスライス定理は、投影データの1次元フーリエ変換が、被写体の2次元フーリエ変換の特定の断面に等しいことを示している。

投影データ $p_\theta(s)$ のフーリエ変換 $P_\theta(\omega)$ は次のように定義される。

\[
P_\theta(\omega) = \int_{-\infty}^{\infty} p_\theta(s) e^{-j2\pi \omega s} \, ds
\]

被写体 $f(x, y)$ の2次元フーリエ変換 $F(u, v)$ は以下の通りである。

\[
F(u, v) = \int_{-\infty}^{\infty} \int_{-\infty}^{\infty} f(x, y) e^{-j2\pi (ux + vy)} \, dx \, dy
\]

フーリエスライス定理は、次の関係を示す。

\[
P_\theta(\omega) = F(u, v) \quad \text{ただし} \quad u = \omega \cos\theta, \quad v = \omega \sin\theta
\]

つまり、投影データのフーリエ変換 $P_\theta(\omega)$ は、被写体の2次元フーリエ空間における角度 $\theta$ の直線(スライス)上の値に等しい。

\section*{3. フーリエスライス定理の証明}

\begin{align*}
P_\theta(\omega) &= \int_{-\infty}^{\infty} p_\theta(s) e^{-j2\pi \omega s} \, ds \\
&= \int_{-\infty}^{\infty} \left( \int_{-\infty}^{\infty} \int_{-\infty}^{\infty} f(x, y) \delta(s - x\cos\theta - y\sin\theta) \, dx \, dy \right) e^{-j2\pi \omega s} \, ds \\
&= \int_{-\infty}^{\infty} \int_{-\infty}^{\infty} f(x, y) \left( \int_{-\infty}^{\infty} \delta(s - x\cos\theta - y\sin\theta) e^{-j2\pi \omega s} \, ds \right) dx \, dy \\
&= \int_{-\infty}^{\infty} \int_{-\infty}^{\infty} f(x, y) e^{-j2\pi \omega (x\cos\theta + y\sin\theta)} \, dx \, dy \\
&= F(\omega \cos\theta, \omega \sin\theta)
\end{align*}

最後の等式は、2次元フーリエ変換 $F(u, v)$ において $u = \omega \cos\theta$, $v = \omega \sin\theta$ と置いた結果である。

\section*{4. CT画像の再構成}

CT画像を再構成する手順は以下の通りである。

\begin{enumerate}
    \item \textbf{投影データの取得}:様々な角度 $\theta$ での投影データ $p_\theta(s)$ を収集する。
    \item \textbf{フーリエ変換の実施}:各投影データ $p_\theta(s)$ に対して1次元フーリエ変換 $P_\theta(\omega)$ を計算します。
    \item \textbf{周波数空間の構築}:フーリエスライス定理に基づき、得られた $P_\theta(\omega)$ を2次元周波数空間 $F(u, v)$ の対応する直線上に配置する。
    \item \textbf{逆フーリエ変換}:周波数空間 $F(u, v)$ に対して2次元逆フーリエ変換を行い、被写体の断面像 $f(x, y)$ を再構成する。
\end{enumerate}

\section*{5. 参考文献}
\verb|http://racco.mikeneko.jp/Kougi/2011a/IPPR/2011aippr12.pdf|

\end{document}
